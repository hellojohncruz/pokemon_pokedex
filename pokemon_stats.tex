% Options for packages loaded elsewhere
\PassOptionsToPackage{unicode}{hyperref}
\PassOptionsToPackage{hyphens}{url}
\PassOptionsToPackage{dvipsnames,svgnames,x11names}{xcolor}
%
\documentclass[
]{article}
\usepackage{amsmath,amssymb}
\usepackage{lmodern}
\usepackage{iftex}
\ifPDFTeX
  \usepackage[T1]{fontenc}
  \usepackage[utf8]{inputenc}
  \usepackage{textcomp} % provide euro and other symbols
\else % if luatex or xetex
  \usepackage{unicode-math}
  \defaultfontfeatures{Scale=MatchLowercase}
  \defaultfontfeatures[\rmfamily]{Ligatures=TeX,Scale=1}
\fi
% Use upquote if available, for straight quotes in verbatim environments
\IfFileExists{upquote.sty}{\usepackage{upquote}}{}
\IfFileExists{microtype.sty}{% use microtype if available
  \usepackage[]{microtype}
  \UseMicrotypeSet[protrusion]{basicmath} % disable protrusion for tt fonts
}{}
\makeatletter
\@ifundefined{KOMAClassName}{% if non-KOMA class
  \IfFileExists{parskip.sty}{%
    \usepackage{parskip}
  }{% else
    \setlength{\parindent}{0pt}
    \setlength{\parskip}{6pt plus 2pt minus 1pt}}
}{% if KOMA class
  \KOMAoptions{parskip=half}}
\makeatother
\usepackage{xcolor}
\usepackage[margin=1in]{geometry}
\usepackage{color}
\usepackage{fancyvrb}
\newcommand{\VerbBar}{|}
\newcommand{\VERB}{\Verb[commandchars=\\\{\}]}
\DefineVerbatimEnvironment{Highlighting}{Verbatim}{commandchars=\\\{\}}
% Add ',fontsize=\small' for more characters per line
\usepackage{framed}
\definecolor{shadecolor}{RGB}{248,248,248}
\newenvironment{Shaded}{\begin{snugshade}}{\end{snugshade}}
\newcommand{\AlertTok}[1]{\textcolor[rgb]{0.94,0.16,0.16}{#1}}
\newcommand{\AnnotationTok}[1]{\textcolor[rgb]{0.56,0.35,0.01}{\textbf{\textit{#1}}}}
\newcommand{\AttributeTok}[1]{\textcolor[rgb]{0.77,0.63,0.00}{#1}}
\newcommand{\BaseNTok}[1]{\textcolor[rgb]{0.00,0.00,0.81}{#1}}
\newcommand{\BuiltInTok}[1]{#1}
\newcommand{\CharTok}[1]{\textcolor[rgb]{0.31,0.60,0.02}{#1}}
\newcommand{\CommentTok}[1]{\textcolor[rgb]{0.56,0.35,0.01}{\textit{#1}}}
\newcommand{\CommentVarTok}[1]{\textcolor[rgb]{0.56,0.35,0.01}{\textbf{\textit{#1}}}}
\newcommand{\ConstantTok}[1]{\textcolor[rgb]{0.00,0.00,0.00}{#1}}
\newcommand{\ControlFlowTok}[1]{\textcolor[rgb]{0.13,0.29,0.53}{\textbf{#1}}}
\newcommand{\DataTypeTok}[1]{\textcolor[rgb]{0.13,0.29,0.53}{#1}}
\newcommand{\DecValTok}[1]{\textcolor[rgb]{0.00,0.00,0.81}{#1}}
\newcommand{\DocumentationTok}[1]{\textcolor[rgb]{0.56,0.35,0.01}{\textbf{\textit{#1}}}}
\newcommand{\ErrorTok}[1]{\textcolor[rgb]{0.64,0.00,0.00}{\textbf{#1}}}
\newcommand{\ExtensionTok}[1]{#1}
\newcommand{\FloatTok}[1]{\textcolor[rgb]{0.00,0.00,0.81}{#1}}
\newcommand{\FunctionTok}[1]{\textcolor[rgb]{0.00,0.00,0.00}{#1}}
\newcommand{\ImportTok}[1]{#1}
\newcommand{\InformationTok}[1]{\textcolor[rgb]{0.56,0.35,0.01}{\textbf{\textit{#1}}}}
\newcommand{\KeywordTok}[1]{\textcolor[rgb]{0.13,0.29,0.53}{\textbf{#1}}}
\newcommand{\NormalTok}[1]{#1}
\newcommand{\OperatorTok}[1]{\textcolor[rgb]{0.81,0.36,0.00}{\textbf{#1}}}
\newcommand{\OtherTok}[1]{\textcolor[rgb]{0.56,0.35,0.01}{#1}}
\newcommand{\PreprocessorTok}[1]{\textcolor[rgb]{0.56,0.35,0.01}{\textit{#1}}}
\newcommand{\RegionMarkerTok}[1]{#1}
\newcommand{\SpecialCharTok}[1]{\textcolor[rgb]{0.00,0.00,0.00}{#1}}
\newcommand{\SpecialStringTok}[1]{\textcolor[rgb]{0.31,0.60,0.02}{#1}}
\newcommand{\StringTok}[1]{\textcolor[rgb]{0.31,0.60,0.02}{#1}}
\newcommand{\VariableTok}[1]{\textcolor[rgb]{0.00,0.00,0.00}{#1}}
\newcommand{\VerbatimStringTok}[1]{\textcolor[rgb]{0.31,0.60,0.02}{#1}}
\newcommand{\WarningTok}[1]{\textcolor[rgb]{0.56,0.35,0.01}{\textbf{\textit{#1}}}}
\usepackage{longtable,booktabs,array}
\usepackage{calc} % for calculating minipage widths
% Correct order of tables after \paragraph or \subparagraph
\usepackage{etoolbox}
\makeatletter
\patchcmd\longtable{\par}{\if@noskipsec\mbox{}\fi\par}{}{}
\makeatother
% Allow footnotes in longtable head/foot
\IfFileExists{footnotehyper.sty}{\usepackage{footnotehyper}}{\usepackage{footnote}}
\makesavenoteenv{longtable}
\usepackage{graphicx}
\makeatletter
\def\maxwidth{\ifdim\Gin@nat@width>\linewidth\linewidth\else\Gin@nat@width\fi}
\def\maxheight{\ifdim\Gin@nat@height>\textheight\textheight\else\Gin@nat@height\fi}
\makeatother
% Scale images if necessary, so that they will not overflow the page
% margins by default, and it is still possible to overwrite the defaults
% using explicit options in \includegraphics[width, height, ...]{}
\setkeys{Gin}{width=\maxwidth,height=\maxheight,keepaspectratio}
% Set default figure placement to htbp
\makeatletter
\def\fps@figure{htbp}
\makeatother
\setlength{\emergencystretch}{3em} % prevent overfull lines
\providecommand{\tightlist}{%
  \setlength{\itemsep}{0pt}\setlength{\parskip}{0pt}}
\setcounter{secnumdepth}{-\maxdimen} % remove section numbering
\ifLuaTeX
  \usepackage{selnolig}  % disable illegal ligatures
\fi
\IfFileExists{bookmark.sty}{\usepackage{bookmark}}{\usepackage{hyperref}}
\IfFileExists{xurl.sty}{\usepackage{xurl}}{} % add URL line breaks if available
\urlstyle{same} % disable monospaced font for URLs
\hypersetup{
  pdftitle={Pokemon},
  pdfauthor={John Cruz},
  colorlinks=true,
  linkcolor={Maroon},
  filecolor={Maroon},
  citecolor={Blue},
  urlcolor={blue},
  pdfcreator={LaTeX via pandoc}}

\title{Pokemon}
\author{John Cruz}
\date{2023-03-01}

\begin{document}
\maketitle

\hypertarget{introduction}{%
\subsection{Introduction}\label{introduction}}

The Serebii website provides a list, called the National Pokedex, of all
the Pokemon in all the games. The table breaks down each Pokemon with a
unique ID number, name, type, abilities and base stats such as attack
and defense.

\textbf{Data Sources:}
\href{https://www.serebii.net/pokemon/nationalpokedex.shtml}{Pokemon
Stats}

Lets determine a frequency chart of which Pokemon fall into which types
they are.

\hypertarget{required-libraries}{%
\subsection{Required Libraries}\label{required-libraries}}

\begin{Shaded}
\begin{Highlighting}[]
\FunctionTok{library}\NormalTok{(tidyverse)}
\FunctionTok{library}\NormalTok{(rvest)}
\FunctionTok{library}\NormalTok{(xml2)}
\FunctionTok{library}\NormalTok{(janitor)}
\end{Highlighting}
\end{Shaded}

\begin{center}\rule{0.5\linewidth}{0.5pt}\end{center}

\hypertarget{web-scrape-table-into-data-frame}{%
\subsection{Web Scrape Table into Data
Frame}\label{web-scrape-table-into-data-frame}}

At first, when the table was web scraped, it brought in the values into
a Data Frame, however, I quickly realized that the abilities listed
multiple ones and the \emph{rvest()} package cleaned it up by adding a
space between them. This causes issues as some abilities are multiple
words so to string manipulate them, I needed to account for the built-in
HTML line breaks (). To do this, I used XML to add an intentional new
line character (\n), so that I could be able to break them apart later.

\href{https://stackoverflow.com/questions/30921626/can-rvest-keep-inline-html-tags-such-as-br-using-html-table}{Using
XML}

\begin{Shaded}
\begin{Highlighting}[]
\NormalTok{url }\OtherTok{\textless{}{-}} \StringTok{"https://www.serebii.net/pokemon/nationalpokedex.shtml"}

\NormalTok{web\_table }\OtherTok{\textless{}{-}} \FunctionTok{read\_html}\NormalTok{(url) }

\CommentTok{\# use XML to account for \textless{}br\textgreater{} with abilities and add \textquotesingle{}\textbackslash{}n\textquotesingle{}}
\FunctionTok{xml\_find\_all}\NormalTok{(web\_table, }\StringTok{".//br"}\NormalTok{) }\SpecialCharTok{|\textgreater{}} 
  \FunctionTok{xml\_add\_sibling}\NormalTok{(}\StringTok{"p"}\NormalTok{, }\StringTok{"}\SpecialCharTok{\textbackslash{}n}\StringTok{"}\NormalTok{)}
\FunctionTok{xml\_find\_all}\NormalTok{(web\_table, }\StringTok{".//br"}\NormalTok{) }\SpecialCharTok{|\textgreater{}} 
  \FunctionTok{xml\_remove}\NormalTok{() }

\NormalTok{web\_table }\OtherTok{\textless{}{-}} 
\NormalTok{  web\_table }\SpecialCharTok{|\textgreater{}} 
  \FunctionTok{html\_element}\NormalTok{(}\StringTok{\textquotesingle{}.dextable\textquotesingle{}}\NormalTok{) }\SpecialCharTok{|\textgreater{}} 
  \FunctionTok{html\_table}\NormalTok{()}

\NormalTok{pokemon\_stats }\OtherTok{\textless{}{-}} \FunctionTok{as.data.frame}\NormalTok{(web\_table)}

\NormalTok{knitr}\SpecialCharTok{::}\FunctionTok{kable}\NormalTok{(}\FunctionTok{head}\NormalTok{(pokemon\_stats))}
\end{Highlighting}
\end{Shaded}

\begin{longtable}[]{@{}
  >{\raggedright\arraybackslash}p{(\columnwidth - 22\tabcolsep) * \real{0.0603}}
  >{\raggedright\arraybackslash}p{(\columnwidth - 22\tabcolsep) * \real{0.0345}}
  >{\raggedright\arraybackslash}p{(\columnwidth - 22\tabcolsep) * \real{0.0431}}
  >{\raggedright\arraybackslash}p{(\columnwidth - 22\tabcolsep) * \real{0.0862}}
  >{\raggedright\arraybackslash}p{(\columnwidth - 22\tabcolsep) * \real{0.0862}}
  >{\raggedright\arraybackslash}p{(\columnwidth - 22\tabcolsep) * \real{0.1810}}
  >{\raggedright\arraybackslash}p{(\columnwidth - 22\tabcolsep) * \real{0.0948}}
  >{\raggedright\arraybackslash}p{(\columnwidth - 22\tabcolsep) * \real{0.0948}}
  >{\raggedright\arraybackslash}p{(\columnwidth - 22\tabcolsep) * \real{0.0948}}
  >{\raggedright\arraybackslash}p{(\columnwidth - 22\tabcolsep) * \real{0.0948}}
  >{\raggedright\arraybackslash}p{(\columnwidth - 22\tabcolsep) * \real{0.0948}}
  >{\raggedleft\arraybackslash}p{(\columnwidth - 22\tabcolsep) * \real{0.0345}}@{}}
\toprule()
\begin{minipage}[b]{\linewidth}\raggedright
X1
\end{minipage} & \begin{minipage}[b]{\linewidth}\raggedright
X2
\end{minipage} & \begin{minipage}[b]{\linewidth}\raggedright
X3
\end{minipage} & \begin{minipage}[b]{\linewidth}\raggedright
X4
\end{minipage} & \begin{minipage}[b]{\linewidth}\raggedright
X5
\end{minipage} & \begin{minipage}[b]{\linewidth}\raggedright
X6
\end{minipage} & \begin{minipage}[b]{\linewidth}\raggedright
X7
\end{minipage} & \begin{minipage}[b]{\linewidth}\raggedright
X8
\end{minipage} & \begin{minipage}[b]{\linewidth}\raggedright
X9
\end{minipage} & \begin{minipage}[b]{\linewidth}\raggedright
X10
\end{minipage} & \begin{minipage}[b]{\linewidth}\raggedright
X11
\end{minipage} & \begin{minipage}[b]{\linewidth}\raggedleft
X12
\end{minipage} \\
\midrule()
\endhead
No. & Pic & Name & Type & Abilities & Base Stats & Base Stats & Base
Stats & Base Stats & Base Stats & Base Stats & NA \\
No. & Pic & Name & Type & Abilities & HP & Att & Def & S.Att & S.Def &
Spd & NA \\
\#00001 & & & Bulbasaur & & Overgrow & & & & & & \\
Chlorophyll & 45 & 49 & 49 & 65 & 65 & 45 & & & & & \\
& NA & NA & NA & NA & NA & NA & NA & NA & NA & NA & NA \\
\#00002 & & & Ivysaur & & Overgrow & & & & & & \\
Chlorophyll & 60 & 62 & 63 & 80 & 80 & 60 & & & & & \\
& NA & NA & NA & NA & NA & NA & NA & NA & NA & NA & NA \\
\bottomrule()
\end{longtable}

\begin{center}\rule{0.5\linewidth}{0.5pt}\end{center}

\hypertarget{tidy-up-data}{%
\subsection{Tidy Up Data}\label{tidy-up-data}}

Every other row that was brought into the Data Frame had \emph{N/A}
values. I removed these rows where if any Pokemon didn't have a name, it
was dropped. The columns between \emph{Name} and the \emph{Base Stats}
also were shifted to the right by one column. Using a vector, I moved
the columns to the appropriate alignment and followed it by using the
\emph{janiotr} package to clean up column names. To account for multiple
abilities, the data frame was then unpivoted into a longer format for
each Pokemon.

\textbf{Note:} Pokemon types are missing and the column was dropped. It
will be explained in the following section.

\begin{Shaded}
\begin{Highlighting}[]
\CommentTok{\# drop null values if Pokemon name is N/A}
\NormalTok{stats\_df }\OtherTok{\textless{}{-}} 
\NormalTok{  pokemon\_stats }\SpecialCharTok{|\textgreater{}} 
  \FunctionTok{drop\_na}\NormalTok{(}\DecValTok{4}\NormalTok{)}

\CommentTok{\# drop first row (duplicate header) and second column (pic)}
\NormalTok{stats\_df }\OtherTok{\textless{}{-}}\NormalTok{ stats\_df[}\SpecialCharTok{{-}}\DecValTok{1}\NormalTok{,}\SpecialCharTok{{-}}\DecValTok{2}\NormalTok{]}

\CommentTok{\# set column headers from first row and clean names}
\NormalTok{stats\_df }\OtherTok{\textless{}{-}} 
\NormalTok{  stats\_df }\SpecialCharTok{|\textgreater{}} 
  \FunctionTok{row\_to\_names}\NormalTok{(}\AttributeTok{row\_number =} \DecValTok{1}\NormalTok{) }\SpecialCharTok{|\textgreater{}} 
  \FunctionTok{clean\_names}\NormalTok{()}

\CommentTok{\# shift pokemon names, etc to left by 1 column}
\NormalTok{stats\_df[}\FunctionTok{c}\NormalTok{(}\DecValTok{2}\SpecialCharTok{:}\DecValTok{10}\NormalTok{)] }\OtherTok{=}\NormalTok{ stats\_df[, }\FunctionTok{c}\NormalTok{(}\DecValTok{3}\SpecialCharTok{:}\DecValTok{11}\NormalTok{)]}

\CommentTok{\# drop \textquotesingle{}na\textquotesingle{} column}
\NormalTok{stats\_df }\OtherTok{\textless{}{-}} 
\NormalTok{  stats\_df }\SpecialCharTok{|\textgreater{}} 
  \FunctionTok{select}\NormalTok{(}\SpecialCharTok{!}\FunctionTok{c}\NormalTok{(na, type))}

\CommentTok{\# split multiple abilities into long format based on created \textquotesingle{}\textbackslash{}n\textquotesingle{}}
\NormalTok{stats\_df }\OtherTok{\textless{}{-}} 
\NormalTok{  stats\_df }\SpecialCharTok{|\textgreater{}} 
  \FunctionTok{separate\_longer\_delim}\NormalTok{(abilities, }\AttributeTok{delim =} \StringTok{"}\SpecialCharTok{\textbackslash{}n}\StringTok{"}\NormalTok{)}

\CommentTok{\# change to pokemon number}
\NormalTok{stats\_df}\SpecialCharTok{$}\NormalTok{no }\OtherTok{\textless{}{-}}
  \FunctionTok{parse\_number}\NormalTok{(stats\_df}\SpecialCharTok{$}\NormalTok{no)}

\NormalTok{knitr}\SpecialCharTok{::}\FunctionTok{kable}\NormalTok{(}\FunctionTok{head}\NormalTok{(stats\_df))}
\end{Highlighting}
\end{Shaded}

\begin{longtable}[]{@{}rlllllllr@{}}
\toprule()
no & name & abilities & hp & att & def & s\_att & s\_def & spd \\
\midrule()
\endhead
1 & Bulbasaur & Overgrow & 45 & 49 & 49 & 65 & 65 & 45 \\
1 & Bulbasaur & Chlorophyll & 45 & 49 & 49 & 65 & 65 & 45 \\
2 & Ivysaur & Overgrow & 60 & 62 & 63 & 80 & 80 & 60 \\
2 & Ivysaur & Chlorophyll & 60 & 62 & 63 & 80 & 80 & 60 \\
3 & Venusaur & Overgrow & 80 & 82 & 83 & 100 & 100 & 80 \\
3 & Venusaur & Chlorophyll & 80 & 82 & 83 & 100 & 100 & 80 \\
\bottomrule()
\end{longtable}

\begin{center}\rule{0.5\linewidth}{0.5pt}\end{center}

\hypertarget{pokemon-types-missing}{%
\subsection{Pokemon Types Missing}\label{pokemon-types-missing}}

While working on the Data Frame, I realized the Pokemon types were
missing. Looking into the HTML code, the types were not text, but images
with no conventional tag name for each one. Beause of this, I used
another website that provides that information and web scraped similarly
into another Data Frame.

\begin{Shaded}
\begin{Highlighting}[]
\NormalTok{url }\OtherTok{\textless{}{-}} \StringTok{"https://bulbapedia.bulbagarden.net/wiki/List\_of\_Pok\%C3\%A9mon\_by\_National\_Pok\%C3\%A9dex\_number"}

\NormalTok{web\_table }\OtherTok{\textless{}{-}} \FunctionTok{read\_html}\NormalTok{(url) }

\CommentTok{\# use XML to account for \textless{}br\textgreater{} and replace with \textquotesingle{}\textbackslash{}n\textquotesingle{}}
\FunctionTok{xml\_find\_all}\NormalTok{(web\_table, }\StringTok{".//br"}\NormalTok{) }\SpecialCharTok{|\textgreater{}} 
  \FunctionTok{xml\_add\_sibling}\NormalTok{(}\StringTok{"p"}\NormalTok{, }\StringTok{"}\SpecialCharTok{\textbackslash{}n}\StringTok{"}\NormalTok{)}
\FunctionTok{xml\_find\_all}\NormalTok{(web\_table, }\StringTok{".//br"}\NormalTok{) }\SpecialCharTok{|\textgreater{}} 
  \FunctionTok{xml\_remove}\NormalTok{() }

\NormalTok{web\_table }\OtherTok{\textless{}{-}} 
\NormalTok{  web\_table }\SpecialCharTok{|\textgreater{}} 
  \FunctionTok{html\_element}\NormalTok{(}\StringTok{\textquotesingle{}body\textquotesingle{}}\NormalTok{) }\SpecialCharTok{|\textgreater{}} 
  \FunctionTok{html\_table}\NormalTok{()}

\NormalTok{pokemon\_types }\OtherTok{\textless{}{-}} \FunctionTok{as.data.frame}\NormalTok{(web\_table)}

\NormalTok{knitr}\SpecialCharTok{::}\FunctionTok{kable}\NormalTok{(}\FunctionTok{head}\NormalTok{(pokemon\_types))}
\end{Highlighting}
\end{Shaded}

\begin{longtable}[]{@{}
  >{\raggedright\arraybackslash}p{(\columnwidth - 74\tabcolsep) * \real{0.1436}}
  >{\raggedright\arraybackslash}p{(\columnwidth - 74\tabcolsep) * \real{0.0166}}
  >{\raggedright\arraybackslash}p{(\columnwidth - 74\tabcolsep) * \real{0.0608}}
  >{\raggedright\arraybackslash}p{(\columnwidth - 74\tabcolsep) * \real{0.0331}}
  >{\raggedright\arraybackslash}p{(\columnwidth - 74\tabcolsep) * \real{0.0387}}
  >{\raggedright\arraybackslash}p{(\columnwidth - 74\tabcolsep) * \real{0.0166}}
  >{\raggedright\arraybackslash}p{(\columnwidth - 74\tabcolsep) * \real{0.0166}}
  >{\raggedright\arraybackslash}p{(\columnwidth - 74\tabcolsep) * \real{0.0166}}
  >{\raggedright\arraybackslash}p{(\columnwidth - 74\tabcolsep) * \real{0.0166}}
  >{\raggedright\arraybackslash}p{(\columnwidth - 74\tabcolsep) * \real{0.0221}}
  >{\raggedright\arraybackslash}p{(\columnwidth - 74\tabcolsep) * \real{0.0221}}
  >{\raggedright\arraybackslash}p{(\columnwidth - 74\tabcolsep) * \real{0.0221}}
  >{\raggedright\arraybackslash}p{(\columnwidth - 74\tabcolsep) * \real{0.0221}}
  >{\raggedright\arraybackslash}p{(\columnwidth - 74\tabcolsep) * \real{0.0221}}
  >{\raggedright\arraybackslash}p{(\columnwidth - 74\tabcolsep) * \real{0.0221}}
  >{\raggedright\arraybackslash}p{(\columnwidth - 74\tabcolsep) * \real{0.0221}}
  >{\raggedright\arraybackslash}p{(\columnwidth - 74\tabcolsep) * \real{0.0221}}
  >{\raggedright\arraybackslash}p{(\columnwidth - 74\tabcolsep) * \real{0.0221}}
  >{\raggedright\arraybackslash}p{(\columnwidth - 74\tabcolsep) * \real{0.0221}}
  >{\raggedright\arraybackslash}p{(\columnwidth - 74\tabcolsep) * \real{0.0221}}
  >{\raggedright\arraybackslash}p{(\columnwidth - 74\tabcolsep) * \real{0.0221}}
  >{\raggedright\arraybackslash}p{(\columnwidth - 74\tabcolsep) * \real{0.0221}}
  >{\raggedright\arraybackslash}p{(\columnwidth - 74\tabcolsep) * \real{0.0221}}
  >{\raggedright\arraybackslash}p{(\columnwidth - 74\tabcolsep) * \real{0.0221}}
  >{\raggedright\arraybackslash}p{(\columnwidth - 74\tabcolsep) * \real{0.0221}}
  >{\raggedright\arraybackslash}p{(\columnwidth - 74\tabcolsep) * \real{0.0221}}
  >{\raggedright\arraybackslash}p{(\columnwidth - 74\tabcolsep) * \real{0.0221}}
  >{\raggedright\arraybackslash}p{(\columnwidth - 74\tabcolsep) * \real{0.0221}}
  >{\raggedright\arraybackslash}p{(\columnwidth - 74\tabcolsep) * \real{0.0221}}
  >{\raggedright\arraybackslash}p{(\columnwidth - 74\tabcolsep) * \real{0.0221}}
  >{\raggedright\arraybackslash}p{(\columnwidth - 74\tabcolsep) * \real{0.0221}}
  >{\raggedright\arraybackslash}p{(\columnwidth - 74\tabcolsep) * \real{0.0221}}
  >{\raggedright\arraybackslash}p{(\columnwidth - 74\tabcolsep) * \real{0.0221}}
  >{\raggedright\arraybackslash}p{(\columnwidth - 74\tabcolsep) * \real{0.0221}}
  >{\raggedright\arraybackslash}p{(\columnwidth - 74\tabcolsep) * \real{0.0221}}
  >{\raggedright\arraybackslash}p{(\columnwidth - 74\tabcolsep) * \real{0.0221}}
  >{\raggedright\arraybackslash}p{(\columnwidth - 74\tabcolsep) * \real{0.0221}}
  >{\raggedright\arraybackslash}p{(\columnwidth - 74\tabcolsep) * \real{0.0221}}@{}}
\toprule()
\begin{minipage}[b]{\linewidth}\raggedright
X1
\end{minipage} & \begin{minipage}[b]{\linewidth}\raggedright
X2
\end{minipage} & \begin{minipage}[b]{\linewidth}\raggedright
X3
\end{minipage} & \begin{minipage}[b]{\linewidth}\raggedright
X4
\end{minipage} & \begin{minipage}[b]{\linewidth}\raggedright
X5
\end{minipage} & \begin{minipage}[b]{\linewidth}\raggedright
X6
\end{minipage} & \begin{minipage}[b]{\linewidth}\raggedright
X7
\end{minipage} & \begin{minipage}[b]{\linewidth}\raggedright
X8
\end{minipage} & \begin{minipage}[b]{\linewidth}\raggedright
X9
\end{minipage} & \begin{minipage}[b]{\linewidth}\raggedright
X10
\end{minipage} & \begin{minipage}[b]{\linewidth}\raggedright
X11
\end{minipage} & \begin{minipage}[b]{\linewidth}\raggedright
X12
\end{minipage} & \begin{minipage}[b]{\linewidth}\raggedright
X13
\end{minipage} & \begin{minipage}[b]{\linewidth}\raggedright
X14
\end{minipage} & \begin{minipage}[b]{\linewidth}\raggedright
X15
\end{minipage} & \begin{minipage}[b]{\linewidth}\raggedright
X16
\end{minipage} & \begin{minipage}[b]{\linewidth}\raggedright
X17
\end{minipage} & \begin{minipage}[b]{\linewidth}\raggedright
X18
\end{minipage} & \begin{minipage}[b]{\linewidth}\raggedright
X19
\end{minipage} & \begin{minipage}[b]{\linewidth}\raggedright
X20
\end{minipage} & \begin{minipage}[b]{\linewidth}\raggedright
X21
\end{minipage} & \begin{minipage}[b]{\linewidth}\raggedright
X22
\end{minipage} & \begin{minipage}[b]{\linewidth}\raggedright
X23
\end{minipage} & \begin{minipage}[b]{\linewidth}\raggedright
X24
\end{minipage} & \begin{minipage}[b]{\linewidth}\raggedright
X25
\end{minipage} & \begin{minipage}[b]{\linewidth}\raggedright
X26
\end{minipage} & \begin{minipage}[b]{\linewidth}\raggedright
X27
\end{minipage} & \begin{minipage}[b]{\linewidth}\raggedright
X28
\end{minipage} & \begin{minipage}[b]{\linewidth}\raggedright
X29
\end{minipage} & \begin{minipage}[b]{\linewidth}\raggedright
X30
\end{minipage} & \begin{minipage}[b]{\linewidth}\raggedright
X31
\end{minipage} & \begin{minipage}[b]{\linewidth}\raggedright
X32
\end{minipage} & \begin{minipage}[b]{\linewidth}\raggedright
X33
\end{minipage} & \begin{minipage}[b]{\linewidth}\raggedright
X34
\end{minipage} & \begin{minipage}[b]{\linewidth}\raggedright
X35
\end{minipage} & \begin{minipage}[b]{\linewidth}\raggedright
X36
\end{minipage} & \begin{minipage}[b]{\linewidth}\raggedright
X37
\end{minipage} & \begin{minipage}[b]{\linewidth}\raggedright
X38
\end{minipage} \\
\midrule()
\endhead
Shortcuts & & & & & & & & & & & & & & & & & & & & & & & & & & & & & & &
& & & & & & \\
\bottomrule()
\end{longtable}

Ndex Olddex Natdex \textbar NA \textbar NA \textbar NA \textbar NA
\textbar NA \textbar NA \textbar NA \textbar NA \textbar NA \textbar NA
\textbar NA \textbar NA \textbar NA \textbar NA \textbar NA \textbar NA
\textbar NA \textbar NA \textbar NA \textbar NA \textbar NA \textbar NA
\textbar NA \textbar NA \textbar NA \textbar NA \textbar NA \textbar NA
\textbar NA \textbar NA \textbar NA \textbar NA \textbar NA \textbar NA
\textbar NA \textbar NA \textbar NA \textbar{} \textbar Ndex \textbar MS
\textbar Pokémon \textbar Type \textbar Type \textbar NA \textbar NA
\textbar NA \textbar NA \textbar NA \textbar NA \textbar NA \textbar NA
\textbar NA \textbar NA \textbar NA \textbar NA \textbar NA \textbar NA
\textbar NA \textbar NA \textbar NA \textbar NA \textbar NA \textbar NA
\textbar NA \textbar NA \textbar NA \textbar NA \textbar NA \textbar NA
\textbar NA \textbar NA \textbar NA \textbar NA \textbar NA \textbar NA
\textbar NA \textbar{} \textbar\#0001 \textbar{} \textbar Bulbasaur
\textbar Grass \textbar Poison \textbar NA \textbar NA \textbar NA
\textbar NA \textbar NA \textbar NA \textbar NA \textbar NA \textbar NA
\textbar NA \textbar NA \textbar NA \textbar NA \textbar NA \textbar NA
\textbar NA \textbar NA \textbar NA \textbar NA \textbar NA \textbar NA
\textbar NA \textbar NA \textbar NA \textbar NA \textbar NA \textbar NA
\textbar NA \textbar NA \textbar NA \textbar NA \textbar NA \textbar NA
\textbar{} \textbar\#0002 \textbar{} \textbar Ivysaur \textbar Grass
\textbar Poison \textbar NA \textbar NA \textbar NA \textbar NA
\textbar NA \textbar NA \textbar NA \textbar NA \textbar NA \textbar NA
\textbar NA \textbar NA \textbar NA \textbar NA \textbar NA \textbar NA
\textbar NA \textbar NA \textbar NA \textbar NA \textbar NA \textbar NA
\textbar NA \textbar NA \textbar NA \textbar NA \textbar NA \textbar NA
\textbar NA \textbar NA \textbar NA \textbar NA \textbar NA \textbar{}
\textbar\#0003 \textbar{} \textbar Venusaur \textbar Grass
\textbar Poison \textbar NA \textbar NA \textbar NA \textbar NA
\textbar NA \textbar NA \textbar NA \textbar NA \textbar NA \textbar NA
\textbar NA \textbar NA \textbar NA \textbar NA \textbar NA \textbar NA
\textbar NA \textbar NA \textbar NA \textbar NA \textbar NA \textbar NA
\textbar NA \textbar NA \textbar NA \textbar NA \textbar NA \textbar NA
\textbar NA \textbar NA \textbar NA \textbar NA \textbar NA \textbar{}
\textbar\#0004 \textbar{} \textbar Charmander \textbar Fire
\textbar Fire \textbar NA \textbar NA \textbar NA \textbar NA
\textbar NA \textbar NA \textbar NA \textbar NA \textbar NA \textbar NA
\textbar NA \textbar NA \textbar NA \textbar NA \textbar NA \textbar NA
\textbar NA \textbar NA \textbar NA \textbar NA \textbar NA \textbar NA
\textbar NA \textbar NA \textbar NA \textbar NA \textbar NA \textbar NA
\textbar NA \textbar NA \textbar NA \textbar NA \textbar NA \textbar{}

Since each Pokemon has a unique identifier, I created the data frame to
relate this ID to its types.

\begin{Shaded}
\begin{Highlighting}[]
\CommentTok{\# drop null values if Pokemon name is N/A}
\NormalTok{types\_df }\OtherTok{\textless{}{-}} 
\NormalTok{  pokemon\_types }\SpecialCharTok{|\textgreater{}} 
  \FunctionTok{drop\_na}\NormalTok{(}\DecValTok{2}\NormalTok{)}

\CommentTok{\# drop unnecessary columns}
\NormalTok{types\_df }\OtherTok{\textless{}{-}} 
\NormalTok{  types\_df[, }\DecValTok{1}\SpecialCharTok{:}\DecValTok{5}\NormalTok{]}

\CommentTok{\# set column headers from first row and clean names}
\NormalTok{types\_df }\OtherTok{\textless{}{-}} 
\NormalTok{  types\_df }\SpecialCharTok{|\textgreater{}} 
  \FunctionTok{row\_to\_names}\NormalTok{(}\AttributeTok{row\_number =} \DecValTok{1}\NormalTok{) }\SpecialCharTok{|\textgreater{}} 
  \FunctionTok{clean\_names}\NormalTok{()}

\CommentTok{\# change to pokemon number}
\NormalTok{types\_df}\SpecialCharTok{$}\NormalTok{ndex }\OtherTok{\textless{}{-}}
  \FunctionTok{parse\_number}\NormalTok{(types\_df}\SpecialCharTok{$}\NormalTok{ndex)}

\CommentTok{\# drop N/A or zero (0) while keeping only distinct pokemon numbers}
\NormalTok{types\_df }\OtherTok{\textless{}{-}} 
\NormalTok{  types\_df }\SpecialCharTok{|\textgreater{}} 
  \FunctionTok{drop\_na}\NormalTok{() }\SpecialCharTok{|\textgreater{}} 
  \FunctionTok{filter}\NormalTok{(ndex }\SpecialCharTok{!=} \DecValTok{0}\NormalTok{) }\SpecialCharTok{|\textgreater{}} 
  \FunctionTok{distinct}\NormalTok{(ndex, }\AttributeTok{.keep\_all=}\ConstantTok{TRUE}\NormalTok{)}

\CommentTok{\# within same pokemon number, replace repeated types with N/A}
\NormalTok{types\_df }\OtherTok{\textless{}{-}} 
\NormalTok{  types\_df }\SpecialCharTok{|\textgreater{}} 
  \FunctionTok{mutate}\NormalTok{(}\AttributeTok{type\_2 =} \FunctionTok{if\_else}\NormalTok{(type\_2 }\SpecialCharTok{!=}\NormalTok{ type, type\_2, }\ConstantTok{NA}\NormalTok{)) }\SpecialCharTok{|\textgreater{}} 
  \FunctionTok{select}\NormalTok{(}\SpecialCharTok{{-}}\FunctionTok{c}\NormalTok{(}\DecValTok{2}\NormalTok{)) }\SpecialCharTok{|\textgreater{}} 
  \FunctionTok{rename}\NormalTok{(}\AttributeTok{no =}\NormalTok{ ndex)}

\CommentTok{\# melt both type columns into one column}
\NormalTok{temp1 }\OtherTok{\textless{}{-}} 
\NormalTok{  types\_df }\SpecialCharTok{|\textgreater{}} 
  \FunctionTok{select}\NormalTok{(}\DecValTok{1}\SpecialCharTok{:}\DecValTok{3}\NormalTok{)}

\NormalTok{temp2 }\OtherTok{\textless{}{-}} 
\NormalTok{  types\_df }\SpecialCharTok{|\textgreater{}} 
  \FunctionTok{select}\NormalTok{(}\DecValTok{1}\NormalTok{,}\DecValTok{2}\NormalTok{,}\DecValTok{4}\NormalTok{) }\SpecialCharTok{|\textgreater{}} 
  \FunctionTok{rename}\NormalTok{(}\AttributeTok{type =}\NormalTok{ type\_2)}

\NormalTok{types\_df }\OtherTok{\textless{}{-}} 
\NormalTok{  temp1 }\SpecialCharTok{|\textgreater{}} 
  \FunctionTok{full\_join}\NormalTok{(temp2) }\SpecialCharTok{|\textgreater{}}
  \FunctionTok{drop\_na}\NormalTok{() }\SpecialCharTok{|\textgreater{}} 
  \FunctionTok{select}\NormalTok{(}\SpecialCharTok{!}\NormalTok{pokemon) }\SpecialCharTok{|\textgreater{}} 
  \FunctionTok{arrange}\NormalTok{(no)}

\NormalTok{knitr}\SpecialCharTok{::}\FunctionTok{kable}\NormalTok{(}\FunctionTok{head}\NormalTok{(types\_df))}
\end{Highlighting}
\end{Shaded}

\begin{longtable}[]{@{}rl@{}}
\toprule()
no & type \\
\midrule()
\endhead
1 & Grass \\
1 & Poison \\
2 & Grass \\
2 & Poison \\
3 & Grass \\
3 & Poison \\
\bottomrule()
\end{longtable}

\hypertarget{merge-pokemon-stats-and-types-tables}{%
\subsection{Merge Pokemon Stats and Types
Tables}\label{merge-pokemon-stats-and-types-tables}}

Using a similar way of SQL joins, I merged the Pokemon to its types and
stats.

\begin{Shaded}
\begin{Highlighting}[]
\NormalTok{stats\_types\_df }\OtherTok{\textless{}{-}}  
\NormalTok{  stats\_df }\SpecialCharTok{|\textgreater{}} 
  \FunctionTok{inner\_join}\NormalTok{(types\_df) }\SpecialCharTok{|\textgreater{}} 
  \FunctionTok{relocate}\NormalTok{(type, }\AttributeTok{.after =}\NormalTok{ name)}

\NormalTok{knitr}\SpecialCharTok{::}\FunctionTok{kable}\NormalTok{(}\FunctionTok{head}\NormalTok{(stats\_types\_df))}
\end{Highlighting}
\end{Shaded}

\begin{longtable}[]{@{}rllllllllr@{}}
\toprule()
no & name & type & abilities & hp & att & def & s\_att & s\_def & spd \\
\midrule()
\endhead
1 & Bulbasaur & Grass & Overgrow & 45 & 49 & 49 & 65 & 65 & 45 \\
1 & Bulbasaur & Poison & Overgrow & 45 & 49 & 49 & 65 & 65 & 45 \\
1 & Bulbasaur & Grass & Chlorophyll & 45 & 49 & 49 & 65 & 65 & 45 \\
1 & Bulbasaur & Poison & Chlorophyll & 45 & 49 & 49 & 65 & 65 & 45 \\
2 & Ivysaur & Grass & Overgrow & 60 & 62 & 63 & 80 & 80 & 60 \\
2 & Ivysaur & Poison & Overgrow & 60 & 62 & 63 & 80 & 80 & 60 \\
\bottomrule()
\end{longtable}

\begin{center}\rule{0.5\linewidth}{0.5pt}\end{center}

\hypertarget{pokemon-type-frequency}{%
\subsection{Pokemon Type Frequency}\label{pokemon-type-frequency}}

The frequency chart shows us that \emph{Water, Normal and Grass} types
are the most common among all the Pokemon. One caveat to point out, is
that a Pokemon could have up to two different types. This means that a
Pokemon can appear twice in different categories as a result in here.

\begin{Shaded}
\begin{Highlighting}[]
\NormalTok{grouped\_type }\OtherTok{\textless{}{-}}  
\NormalTok{  stats\_types\_df }\SpecialCharTok{|\textgreater{}} 
  \FunctionTok{distinct}\NormalTok{(name, type) }\SpecialCharTok{|\textgreater{}} 
  \FunctionTok{group\_by}\NormalTok{(type) }\SpecialCharTok{|\textgreater{}} 
  \FunctionTok{summarise}\NormalTok{(}\AttributeTok{count =} \FunctionTok{n}\NormalTok{()) }\SpecialCharTok{|\textgreater{}} 
    \FunctionTok{arrange}\NormalTok{(}\FunctionTok{desc}\NormalTok{(count))}

\NormalTok{grouped\_type }\SpecialCharTok{|\textgreater{}} 
  \FunctionTok{ggplot}\NormalTok{(}\FunctionTok{aes}\NormalTok{(}\AttributeTok{x =}\NormalTok{ count, }\AttributeTok{y =} \FunctionTok{reorder}\NormalTok{(type, count))) }\SpecialCharTok{+}
  \FunctionTok{geom\_bar}\NormalTok{(}\AttributeTok{stat =} \StringTok{\textquotesingle{}identity\textquotesingle{}}\NormalTok{)}
\end{Highlighting}
\end{Shaded}

\includegraphics{pokemon_stats_files/figure-latex/plot-1.pdf}

\begin{Shaded}
\begin{Highlighting}[]
\NormalTok{knitr}\SpecialCharTok{::}\FunctionTok{kable}\NormalTok{(grouped\_type)}
\end{Highlighting}
\end{Shaded}

\begin{longtable}[]{@{}lr@{}}
\toprule()
type & count \\
\midrule()
\endhead
Water & 154 \\
Normal & 130 \\
Grass & 122 \\
Flying & 109 \\
Psychic & 99 \\
Bug & 92 \\
Fire & 80 \\
Poison & 79 \\
Ground & 75 \\
Rock & 73 \\
Fighting & 72 \\
Dark & 69 \\
Electric & 68 \\
Dragon & 65 \\
Fairy & 63 \\
Steel & 63 \\
Ghost & 62 \\
Ice & 48 \\
\bottomrule()
\end{longtable}

\end{document}
